\documentclass{beamer}

\usepackage{amsmath}

\def\inputGnumericTable{}

\usepackage[latin1]{inputenc}
\usepackage{color}
\usepackage{array}
\usepackage{longtable}
\usepackage{calc}
\usepackage{multirow}
\usepackage{hhline}
\usepackage{ifthen}

\usepackage{lscape}

\usetheme{CambridgeUS}

\title{Assignment 6}
\author{Archit Ganvir (CS1BTECH11005)}
\date{\today}
\logo{\large \LaTeX{}}

\begin{document}
\providecommand{\brak}[1]{\ensuremath{\left(#1\right)}}
\begin{frame}

\titlepage

\begin{abstract}
This document gives the solution for Assignment 6 (Papoulis ch.9 Problem 9.10).
\end{abstract}

\end{frame}

\logo{}

\begin{frame}{Question}

(Problem 9.10) Q.) The process x(t) is normal WSS and E\{x(t)\} = O. Show that if z(t) = $x^2$(t), then $C_{zz}(\tau) = 2C_{xx}^2(\tau)$.

\end{frame}

\begin{frame}{Solution}

Solution : We shall show that if x is a normal process with zero mean and z(t) = $x^2$(t), then $C_{zz}(\tau) = 2 C_{xx}^2(\tau)$.

We know that if the random variables $x_i$ are jointly normal with zero mean, and E\{$x_i x_j$\} = $C_{ij}$, then
\begin{align}
E\{x_1 x_2 x_3 x_4\} = C_{12} C_{34} + C_{13} C_{24} + C_{14} C_{23}
\end{align}

\end{frame}

\begin{frame}

Hence, if the R.V.s $x_k$ are normal and E\{$x_k$\} = 0, then
\begin{align}
E\{x_1 x_2 x_3 x_4\} = E\{x_1 x_2\} E\{x_3 x_4\} + E\{x_1 x_3\} E\{x_2 x_4\} + E\{x_1 x_4\} E\{x_2 x_3\}
\end{align}

With $x_1 = x_2 = x(t + \tau)$ and $x_3 = x_4 = x(t)$, we conclude that the autocorrelation of z(t) equals
\begin{align}
E\{x^2(t + \tau)x^2(t)\} &= E^2\{x^2(t + \tau)\} + 2 E^2\{x(t + \tau) x(t)\} \\
&= R_{xx}^2 (0) + 2 R_{xx}^2(\tau)
\end{align}

\end{frame}

\begin{frame}

And since $R_{xx}(\tau) = C_{xx}(\tau)$, and $E\{z(t)\} = R_{xx} (0)$, the above yields
\begin{align}
C_{zz}(\tau) = R_{zz}(\tau) - E^2 \{z(t)\} = 2 C_{xx}^2(\tau)
\end{align}

\end{frame}

\end{document}