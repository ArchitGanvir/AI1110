\documentclass[journal,12pt,twocolumn]{IEEEtran}
\usepackage{tfrupee}
\usepackage[cmex10]{amsmath}
\usepackage{listings}
\usepackage{array}

\begin{document}

\newcommand{\solution}{\noindent \textbf{Solution: }}

\lstset{
%language=C,
frame=single, 
breaklines=true,
columns=fullflexible
}

\title{Assignment 1}
\author{Archit Ganvir (CS21BTECH11005)}

\maketitle

\begin{abstract}
    This document gives the solution for Assignment 1 (ICSE Class 10 Maths 2018 Q.1(b)).
\end{abstract}

Q.) Sonia had a recurring deposit account in a bank and deposited \rupee600 per month for 2½ years. If the rate of interest was 10\% p.a., find the maturity value of this account.\\

\solution
\begin{itemize}

    \item \emph{Given} :-
                \\
                Sum deposited every month, P = 600
                \\
                Number of months, n = number of years x 12 = 2.5 x 12 = 30
                \\
                Rate of interest (p.a.), r = 10\%
                \\
    		\begin{table}[ht!]
			\begin{tabular}{|c|c|c|}
	\hline
	\textbf{Symbol} & \textbf{Value} & \textbf{Description}		    \\
	\hline
	P 			& 600		& Sum deposited every month \\
	\hline
	n 			& 30			& Number of months		    \\
	\hline
	R 			& 10			& Rate of interest (p.a.)	    \\
	\hline
\end{tabular}\\
			\caption{}
		\end{table}
    \item \emph{Formula} :-
                \\
                Total sum deposited = Sum deposited every month x Number of months
                \begin{align}
                    P_{total} = P \times n
                \end{align}
                Interest on the total sum deposited,
                \begin{align}
                    I = P \times \frac{n(n + 1)}{2 \times 12} \times \frac{r}{100}
                \end{align}
                Maturity value of the recurring deposit, M.V. = Total sum deposited + Interest on the total sum deposited
                \begin{align}
                    M.V. = P_{total} + I
                \end{align}
		\begin{table}[ht!]
			\begin{tabular}{|c|p{3.2cm}|}
	\hline
	\textbf{Formula}											    & \textbf{Description}			      \\
	\hline
	$P_{total} = P \times n$										    & Total sum deposited			      \\
	\hline
	$I = P \times \frac{n(n + 1)}{2 \times 12} \times \frac{r}{100}$			    & Interest on the total sum deposited     \\
	\hline
	$M.V. = P_{total} + I$										    & Maturity value of the recurring deposit \\
	\cline{1-1}
	$M.V. = P \times n + P \times \frac{n(n + 1)}{2 \times 12} \times \frac{r}{100}$ &							       \\
	\hline
\end{tabular}\\
			\caption{}
		\end{table}

    \item \emph{To Find} :-
                \\
                The maturity value of the given account
                \\

    \item \emph{Solution} :-
                \\
                From eq.s (1), (2), (3), we get :-
                \begin{align}
                    M.V. = P \times n + P \times \frac{n(n + 1)}{2 \times 12} \times \frac{r}{100}
                \end{align}
                Substituting the values of P, n and r, and solving, we get :-
                \begin{align}
                    M.V. = 20325
                \end{align}

\end{itemize}

Therefore, the maturity value of the account is \rupee20325.\\
\\
The code in
\begin{lstlisting}
Assignment1/codes/recdep.py
\end{lstlisting}
verifies the solution.

\end{document}