\documentclass{beamer}

\usepackage{amsmath}

\def\inputGnumericTable{}

\usepackage[latin1]{inputenc}
\usepackage{color}
\usepackage{array}
\usepackage{longtable}
\usepackage{calc}
\usepackage{multirow}
\usepackage{hhline}
\usepackage{ifthen}

\usepackage{lscape}

\usetheme{CambridgeUS}

\title{Assignment 4}
\author{Archit Ganvir (CS1BTECH11005)}
\date{\today}
\logo{\large \LaTeX{}}

\begin{document}
\providecommand{\brak}[1]{\ensuremath{\left(#1\right)}}
\begin{frame}

\titlepage

\begin{abstract}
This document gives the solution for Assignment 4 (Papoulis ch.2 Example 2-9).
\end{abstract}

\end{frame}

\logo{}

\begin{frame}{Question}

(Example 2-9) Q.) Two players A and B draw balls one at a time alternately from a box containing m white balls and n black balls. Suppose the player who picks the first white ball wins the game. What is the probability that the player who starts the game will win?

\end{frame}

\begin{frame}{Solution}

Solution : We obtain the following distribution of balls in the box :-
\begin{table}[ht!]
	\input{tables/dist_table.tex}
\caption{}
\label{table:table1}

\end{table}

\end{frame}

\begin{frame}

Suppose A starts the game. The game can be won by A if he extracts a white ball at the start or if A and B draw a black ball each and then A draws a white one, or if A and B extract two black balls each and then A draws a white one and so on. Let

$X_k =$ \{A and B alternately draw k black balls each and then A draws a white ball\}\\
where, k = 0, 1, 2, ...

Here, the $X_k$s represent mutually exclusive events and moreover the event
\begin{align}
\{\text{A wins}\} = X_0 \cup X_1 \cup X_2 \cup ...
\end{align}

Hence
\begin{align}
P_A \triangleq P(\text{A wins}) = P(X_0 \cup X_1 \cup X_2 \cup ...) \label{eq:eq1}
\end{align}

\end{frame}

\begin{frame}

The axiom of infinite additivity states that if the events $A_1$, $A_2$, ... are mutually exclusive, then
\begin{align}
P(A_1 \cup A_2 \cup ...) = P(A_1) + P(A_2) + ... \label{eq:eq2}
\end{align}

From eq.s \eqref{eq:eq1}, \eqref{eq:eq2}, we get
\begin{align}
P_A = P(X_0) + P(X_1) + P(X_2) + ...
\end{align}

Now,
\begin{align}
P(X_0) &= \frac{m}{m + n} \\
P(X_1) &= \frac{n}{n + m} . \frac{n - 1}{m + n - 1} . \frac{m}{m + n - 2} \\
&= \frac{n(n - 1)m}{(m + n)(m + n - 1)(m + n - 2)}
\end{align}
and
\begin{align}
P(X_2) = \frac{n(n - 1)(n - 2)(n - 3)m}{(m + n)(m + n - 1)(m + n - 2)(m + n - 3)}
\end{align}

\end{frame}

\begin{frame}
So,
\begin{align}
P_A = \frac{m}{m + n} \brak{1 + \frac{n(n - 1)}{(m + n - 1)(m + n - 2)} + ...}
\end{align}

The code in
\begin{block}{}
Assignment4/codes/prob.py
\end{block}
verifies the solution.

\end{frame}

\end{document}